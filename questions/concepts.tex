\documentclass[11pt]{article}
\usepackage{myanswer}
\begin{document}

\begin{center}
    Yifeng He, 917050499
\end{center}

\begin{center}
    Question 7:
    What is the difference between construction of concepts and analysis of concepts?
\end{center}

Construction of concepts and analysis of concepts are two ideas derived from Kant's idea of concepts,
corresponding to mathematics and philosophy.
These two are two different to reason about concepts.

We have several way to reason about concepts.
Suppose we have a sentence about how subject is predicate.
Synthetic a-posteriori is where predicate is not contained in the subject and get added by experience.
For example, ``All bodies are heavy'',
where the predicate ``heavy'' is based on experience.
On the other hand,
analytic, a-priori is where predicate is contained in the subject before experience by reason.
For example, ``All bachelors are unmarried'',
where the predicate ``unmarried'' is the meaning of the subject ``bachelors''.
And other permutations of synthetic a-priori and analytic a-priori is allowed.

From the above, we can see that the idea of concepts lies between subjects and predicates.
Philosophers analysis the relationship between subjects and predicates.
To construct concepts, we can start with an idea and get additional objects from it.
For example, with the intuition of time,
we can start with a moment and get the next moment, its successor, from it,
which results in the concepts of arithmetics.
On the other hand, with the intuition of space,
we can start with a point, and form a line with the points.
Then the lines can form figures, which results in the concepts of geometry.
\end{document}