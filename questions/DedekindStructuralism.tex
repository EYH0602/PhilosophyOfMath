\documentclass[11pt]{article}
\usepackage{myanswer}
\begin{document}

\begin{center}
    Yifeng He, 917050499
\end{center}

\begin{center}
    Question 10:
    Connect Dedekind's structuralist view of natural numbers to the Hilbert and Bernays quotes 
\end{center}

% Hilbert: We think of ... points, straight lines, and plans as having certain mutual relations, which 
% we indicate by means of such works as ``are situated'', ``between'', ``parallel'', ``congruent'', 
% ``continuous'', etc. The complete and exact description of these relations flows as a consequence 
% of the axioms of geometry.
% 
% Bernays: A main feature of Hilbert's axiomatization of geometry is that the axiomatic 
% method is presented and practiced in the spirit of the abstract conception of axiomatics 
% that arose at the end of the nineteenth century and which has been generally adopted in 
% modern mathematics. It consists in abstracting from the intuitive meaning of the terms for 
% the kinds of primitive objects (individuals) and for the fundamental relations and in 
% understanding the assertions (theorems) of the axiomatized theory in a hypothetical sense, 
% that is, as holding true for any interpretation or determination of the kinds of individuals 
% and of the fundamental relations for which the axioms are satisfied. Thus, an axiom 
% system is regarded not as a system of statements about a subject matter but as a system of 
% conditions for what might be called a relational structure.
% -> a theory gives a relational structure

% The Axioms of Euclidean Plane Geometry
% 1. A straight line may be drawn between any two points.
% 2. Any terminated straight line may be extended indefinitely.
% 3. A circle may be drawn with any given point as center and any given radius.
% 4. All right angles are equal.

% But the fifth axiom was a different sort of statement:

% 5. If two straight lines in a plane are met by another line, 
% and if the sum of the internal angles on one side is less than two right angles, 
% then the straight lines will meet if extended sufficiently on the side on 
% which the sum of the angles is less than two right angles.


% paragraph 1: what is Dedekind's structuralist view of natural numbers
Dedekind's structuralist view of natural numbers is a theory that
natural numbers should be defined as a system in terms of their properties and relationships
as said out in the Peano axioms $P^2A$.
In fact, for Dedekind, such a system is not limited to $\{1, 2, 3, \ldots\}$,
it is whatever satisfies axioms, for example
$\{0, 1, 2, \ldots\}$,
$\{ \emptyset, \{\emptyset\}, \{\{\emptyset\}\}, \ldots \}$,
$\{\lambda f. \lambda x.x, \lambda f. \lambda x.fx, \lambda f. \lambda x.f(fx), \ldots\}$,
or any $\omega$-sequence satisfies the axioms are natural numbers.
I will describe the similarities between Dedekind's structuralist view of natural numbers to the Hilbert and Bernays quotes,
then conclude with some difference between Hilbert's view and Dedekind's view.

% paragraph 2: main connection: variable domain
%   what is Hilbert's view of variable domain in geometry
%   explain Dedekind's structuralist view to variable domain
For Hilbert, from his quote listed, geometry objects
like points, lines, and planes are defined by the relations between them,
such as ``are situated'', ``between'', ``parallel'', ``congruent'', ``continuous'', etc.
These are all relations set by the axioms.
In other words, we implicitly defines the objects in virtual of relations that are given by axioms.
This view gives the notion of \textit{variable domain}.
That is, points, lines, planes can be anything,
even chairs, tables, coffee mugs,
as long as they satisfies the relations given the axioms.
In comparison to Dedekind's view of natural numbers,
Dedekind believed that any system satisfies $P^2A$ is a system of natural numbers.
There two views from Hilbert and Dedekind ares just as described in Bernays quote
``an axiom system is regarded not as a system of statements about a subject matter 
but as a system of conditions for what might be called a relational structure.''

% paragraph 3: theory is _categorical_:
%   all the systems that satisfies the axioms are equivalent up to isomorphism.
%   axioms define *structure* (Hilbert), or they define *systems that have a structure* (Dedekind)
In conclusion to Hilbert's view of geometry and Dedekind's view of natural numbers
is that whether the axioms is \textit{categorical}.
That is, all the systems that satisfies the axioms are equivalent up to isomorphism.
Although similar, there is still a different between these two views.
Hilbert believed that axioms define \textit{structure},
whereas Dedekind believed that axioms define \textit{systems that have a structure}.
Other than this difference,
the same central idea is that an object is defined solely un terms of its relations,
which is (sort of) prove by the Yoneda Lemma.

\end{document}