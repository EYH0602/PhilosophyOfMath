\documentclass[11pt]{article}
\usepackage{myanswer}

\newcommand{\cat}{%
    \mathbf%
}
\newcommand{\domain }[1]{%
    \mathrm{dom}(#1)%
}
\newcommand{\codomain}[1]{%
    \mathrm{cod}(#1)%
}
\newcommand{\idarrow}[1][]{%
    \mathbf{1}{#1}%
}

\begin{document}

\begin{center}
    Yifeng He, 917050499
\end{center}

\begin{center}
    Question 15:
    What are at least two objections to taking category theory as a mathematical foundation?
\end{center}

% what is category theory
% what does it means to take category theory as a mathematical foundation
Category theory (CT) is a two sorted system of objects and arrows such that the axioms are satisfied.
To take something as a mathematical foundation is to give account of structure or system that has a structure
while stays true to the Hilbert/Dedekind theory instead of the Fregean approach.
To achieve, we can either take set theory or category theory as the background theory 
that allows us to talk struct or system.
Taking category theory as a mathematical foundation means that either structures are some kind of category,
or systems are categories.
However, there are many objections to this approach from philosophers like Feferman, Shapiro, and Hellman.

% objection 1: does CT <u>depend on</u> set theory / class theory?
% objection 2: only at object level, at the meta-level you have to ST as a Fregeian foundation.
% objection 3: e if-thenist attempt to stop the regress will end in the problem facing the deductist.
The first problem of taking category theory as a mathematical foundation is the dependency,
or the question of ``does CT \underline{depend on} set theory / class theory?''
Suppose we define CT as a two sorted system with sets as `object' and functions as `arrow', as ETCS.
This objection from Feferman questions weather the intensional notion of set is logically/epistemological/conceptually prior
to the notion of category.
Then we still need set theory as a foundation.
The objection from Shapiro states that we can be a Hilbertian about CT at the object level but not the meta-level.
In other words, to talk about the meta-level of category, we still need the set theory.
Therefore, at the meta-level we still have to take set theory as a Fregeian foundation.
A problem is on the size of category.
To define the category of categories would causes a chain of regress to set theory, class theory, and universe theory.
To avoid this infinite regress of meta theories, we need to take it on a Fregean foundation.

% As-ifism to solve the objections
To solve these objects, we can use the CCAF definition as the meta theory,
where a category $\cat{Cat}$ is a two sorted system with categories as `objects' and functors as `arrows'.
However, some problem still exists with this CCAF definition.
Since the category has a size problem, the category of all categories $\cat{CAT}$ will not be a category,
which is not well formed.
Then the problem of infinite regress still exists.
Another approach is use As-ifism to cut a midpoint,
which take background not as Fregean theory but as if they were consistent.

\end{document}