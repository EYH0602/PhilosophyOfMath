\documentclass[11pt]{article}
\usepackage{myanswer}
\begin{document}

\begin{center}
    Yifeng He, 917050499
\end{center}

\begin{center}
    Question 9:
    What was the Frege/Hilbert debate and what difference did it reveal between Frege and Hilbert?
\end{center}

% what is the Frege/Hilbert debate
The Frege Hilbert debate was a controversy about the nature of geometry.
Gottlob Frege viewed geometry as purely logical concepts.
He believed that the definition of 
geometric objects were constructed from the intuition of space.
Then the definitions given out the axioms,
and sentences about geometry follow from the definitions and axioms by proof.
Based on this believe,
Frege argued that mathematical objects,
such as point, line, and plane,
existed independently of out knowledge of them.
Thus Frege's reasoning followed that
definition/axioms $\rightarrow$ truth $\rightarrow$ consistency.
This view of logicism is later challenged by Russell's Paradox.

Davis Hilbert, on the other hand,
thought that geometry starts from axioms
which implicitly define the things that satisfies them.
Sentences then follows from axioms.
Based on this view,
Hilbert argued that mathematical objects and truths are constructed by mathematicians
and they were ultimately dependent on our ability to manipulate and reason about them.
In other words, whatever satisfies the axioms are mathematical objects,
and they mean whatever the axioms says they mean.
Therefore, axioms serves as schema for concepts,
and mathematical objects has variable domains,
or to say they have no fixed meaning.
Hilbert's view of geometry follows
consistency of axioms $\rightarrow$ truth $\rightarrow$ existence,
which implies that consistency of axioms is the foundation of mathematics.
Consistency of axioms system is later proved not exists by G\"odel's inconsistency theorem.

This debate revealed fundamental difference between the two view of mathematics
regarding the nature of mathematical truth
and the role of logic and intuition in mathematics. 
These two different views is later described as Realism and Formalism,
also described the relationship between mathematics and human.

\end{document}