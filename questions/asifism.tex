\documentclass[11pt]{article}
\usepackage{myanswer}
\begin{document}

\begin{center}
    Yifeng He, 917050499
\end{center}

\begin{center}
    Question 6: What is mathematical as-ifism?
\end{center}

% what is mathematical as-ifism
Mathematical as-ifism is to cut in-between realism and non-realism.
Instead of requiring the objects to exists,
as-ifism makes use of stable definition to deduce a conclusion.
As-ifism utilize hypothetical method, which follows that
take our hypothesis \underline{as if} they were true,
make use of \underline{stable} definitions,
make use of some deductive process,
which then result in conclusion.
That is, instead of existence $\rightarrow$ truth,
we assume truth then deduce existence. 

Mathematical realism  believe that mathematical objects are part of the meta-physical realm.
In this position, mathematical objects exists independently.
In this meta-physical real, the way objects are fixes the truth about them,
and we come to know these objects via reasoning through the dialectical method.
Therefore, for statements to hold, the objects in the statements has to exist.
For example, let `2 is even' be true analogous to `the ball is read' is true,
then both the mathematical object `2' and the physical object `ball' has to exists.
From existence, realism then have reference and satisfaction,
then can explain truth in terms of satisfaction.
However, existence is not often easy to detect.
In this case, as-ifism provides a way to get existence from they hypothesis as they were true.
For example, if we only have the the arithmetic definition of numbers (natural numbers but not the real numbers),
$\sqrt{2}$ does not exist in the realm.
But if start with the geometric definition of numbers,
we can deduce that the length of the side that will double the area of the square is $2\sqrt{2}$.

In comparison to realism and non-realism,
as-ifism explain satisfaction in terms of truth
rather than explain truth in terms of satisfaction of the predicate.
With the assumption of something being truth,
there must be something making it true.
In this way, we can avoid the problem of requiring existence to derive truth.

\end{document}
