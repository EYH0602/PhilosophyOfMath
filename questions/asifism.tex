\documentclass[11pt]{article}
\usepackage{myanswer}
\begin{document}

\begin{center}
    Yifeng He, 917050499
\end{center}

\begin{center}
    Question 6: What is mathematical as-ifism?
\end{center}

% what is mathematical as-ifism
Mathematical as-ifism is to cut in-between realism and non-realism.
Instead of requiring uniform semantics and the objects to exists,
as-ifism makes use of stable definition to deduce a conclusion.
As-ifism utilize hypothetical method, which follows that
take our hypothesis \underline{as if} they were true,
make use of \underline{stable} definitions,
make use of some deductive process,
which then result in conclusion.
That is, instead of existence $\rightarrow$ truth,
we assume truth then deduce existence. 

If we start with the arithmetic definition of number,
there is no way to get an answer to the Meno's problem since $2\sqrt{2}$ does not exists.
But if we start with geometric definition of number,
we are able to find the answer.
Therefore, the notion of number will change depends on its definition.
This example shows that the truth we started with will affect the existence of the object.

In comparison to realism and non-realism,
as-ifism explain satisfaction in terms of truth
rather than explain truth in terms of satisfaction of the predicate.
With the assumption of something being truth,
there must be something making it true.
In this way, we can avoid the problem of uniform semantics.

\end{document}
