\documentclass[11pt]{article}
\usepackage{myanswer}
\begin{document}

\begin{center}
    Yifeng He, 917050499
\end{center}

\begin{center}
    Question 4:
    What is the hypothetical (or mathematician's) method and how does it solve Meno's Paradox?
\end{center}

% what is hypothetical method
The Hypothetical Method, also called Mathematician's method,
is a way of reasoning that only requires stable definitions instead of stable objects.
The hypothetical method is proposed by Plato and follows steps as begin with a hypothesis,
act as if it was a first principle whose truth cannot be doubted,
then reason consistently down to a conclusion with the aim of solving the problem.
The Meno's Paradox questions how will we look for something when we don't know what it is;
and how will we know we've found it when we don't at least know what it is. 
After demonstrating how to solve Meno's Paradox by the hypothetical method, 
I will conclude with some limitations of this method.

% how does it solve Meno's Paradox
To resolve Meno's Paradox,
Plato developed the hypothetical method that
only looks at inconsistency outside its domain or external consistency.
By using mathematician's method,
we can get a kind of knowledge that is secure as any knowledge.
To have knowledge without knowing what something is,
we can just set a hypothesis and see if we can deduce an answer from that hypothesis.
% example, the Meno Demonstration
For example, we know that there exists a length of the side that will double the area of the square.
We don't have to know that length is $2\sqrt{2}$.
Following the mathematical method,
we can begin with the definition of a square.
Suppose there is a square with side length of $2$ units,
then the area of that square is $4$ by definition of the area of the square.
To double the area of the square,
we wish to find the length of the sides that will double the area of the square.
If we double the length of the origional sides,
we will quadruple the area to $16$.
Then with the definition of the diagonal of the square, which will half area of any square,
we can get a square of area $8$ out of the quadrupled square.
Therefore we deduced an answer to the question without knowing that is the actual side length is.

However, the hypothesis method has some limitations.
Deducing from a hypothesis,
we can only yield what is called \textit{true opinion},
not knowledge.
Since this true opinion depends on the hypothesis,
it is not stable,
which means that if the hypothesis is wrong, so is that conclusion.

\end{document}