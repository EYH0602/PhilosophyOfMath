\documentclass[11pt]{article}
\usepackage{myanswer}
\begin{document}

\begin{center}
Yifeng He, 917050499
\end{center}

\begin{center}
Question 12: Explain the difference between \textit{in re} and \textit{ante rem} mathematical structuralism.
\end{center}

% p1: what is mathematical structuralism
% what is \textit{in re} position
% what is \textit{ante rem} position
Mathematical structuralism is a view that takes structures as first-class citizens in philosophy.
It accounts for axioms which define either the structure itself or a system that has a structure.
In this view, an object is just a position in such a structure or system.
There are two positions on mathematical structuralism: \textit{in re} and \textit{ante rem}.
These two positions concern the \textit{priority} of structure to object in mathematics. There are three senses of \textit{priority}:
\begin{enumerate}
    \item ontological sense: structures are ontologically prior to objects (\textit{ante rem}),
    \item semantic sense: systems that have a structure are semantically prior to objects (\textit{in re}),
    \item methodological sense: systems that have a structure are methodologically prior to objects (\textit{in re}).
\end{enumerate}

In general, \textit{ante rem} believes that objects come before structure,
whereas \textit{in re} believes that objects are part of the structure.
To further illustrate the difference between \textit{in re} and \textit{ante rem},
we can take the sentence ``The rose is red'' as an example.
For \textit{in re}, the sentence would be interpreted as redness being in the rose, and we come to know red through individual roses. 
This shows the idea of universals in particulars and that there are only particulars. 
However, for \textit{ante rem}, the sentence would be interpreted as the rose being red because it exemplifies/instantiates Redness, 
which takes universals over and above particulars and types over and above tokens.

% p3 other view
The representatives of \textit{in re} and \textit{ante rem} views of \textit{priority} are Hellman and Shapiro. 
Although these are the two main approaches to understanding 
the semantic sense and ontological sense of priority of structures in mathematics, there are still other views. 
Instead of viewing background theories as giving actual structure or possible systems,
as-ifism acts as if our axioms are true and this allows us to make methodological use of objects for the purpose of solving problems. Therefore, as-ifism uses category theory as a background language to talk about systems that have a structure.

\end{document}