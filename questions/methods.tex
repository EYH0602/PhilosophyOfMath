\documentclass[11pt]{article}
\usepackage{myanswer}
\begin{document}

\begin{center}
    Yifeng He, 917050499
\end{center}

\begin{center}
    Question 5:
    what is the distinction between the hypothetical and the dialectical method?
\end{center}

The difference between the hypothetical and the dialectical method is the
idea of reason and understanding. The hypothetical method is a method of
mathematicians, which starts with a set of hypotheses, acts as if it was
a first principle (whose truth cannot be doubted), then reason
consistently down to a conclusion. Instead of taking a hypothesis as a first
principle, the dialectical method begins with a hypothesis as a
hypothesis, then reasons up to a first principle. Once we reach the first
principle in the dialectical method, we can then reason down to a
conclusion. The key difference is that mathematicians only need stable
definitions, but philosophers need stable objects.

There is also a difference in the usage of the conclusion in these two
methods. In the hypothetical method, the conclusion is used to solve a
problem, but in the dialectical method, the conclusion is used to
justify a true opinion. The true opinion is justified by tagging to a
first principle, which is a form. Therefore, mathematicians use the
hypothetical method to solve a problem. If the problem changes, the
hypothesis is also going to change. For example, there are two competing
accounts of numbers: the Pythagoreans which take the number as units, and the
arithmetics view. We can use one account to solve a problem and one to
solve another. Therefore, there is no connection between the two, and
the hypothetical is not a part of the dialectical method.

One way to cut between the two is As-ifism. From Plato's account of
the hypothetical method, we take hypothesis \textit{as if} they were the first
principle. In as-ifism, we take hypothesis \textit{as if} they were true, and then
we make use of stable definitions in both hypothesis and conclusion to
solve problems. The difference between as-ifism and the hypothetical
method is that as-ifism moves from truth to existence, whereas the
hypothetical method moves from existence to truth.


\end{document}