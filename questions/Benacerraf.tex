\documentclass[11pt]{article}
\usepackage{myanswer}
\begin{document}

\begin{center}
    Yifeng He, 917050499
\end{center}

\begin{center}
    Question 3:
    What is Benacerraf's Dilemma?
\end{center}

% What is Benacerraf's Dilemma?
Benacerraf's dilemma was proposed as a discussion of ``What Numbers Could Not Be'' between
the Realist and Non-Realist position of numbers.
In general, this dilemma can be extent to all mathematical objects.
In Realist's account,
mathematical objects exists independent of us and we came to know those objects directly or indirectly.
For Non-realists,
mathematical objects are invented by us in our minds.
Benacerraf's dilemma points out there is problem with both position.

To be more precise, I will elaborate on the problems with both position.
For Realist, if we want shared uniform semantics, we need mathematical reference.
But if we have mathematical reference, 
then we face the problem of epistemic access.
In other words, if mathematical objects are abstract objects exist independently of us,
we cannot have the access to the information about these abstract objects.
For Non-realists, if we start with reasonable (casual) epistemology,
then there is no mathematical reference, 
thus we will not have uniform semantics.
If we invented these objects in our minds,
we cannot have subjection knowledge about them.

Although there is no direct solution to Benacerraf's Dilemma,
one way proposed to `bypass' the dilemma is Structuralism.
Instead of focusing on the objects itself,
we can focus on what is the structure.
In this way, we can define numbers as having the structure of Piano Axioms.
However, there is also debates on the view of Structuralism.
Derived from Structuralism's position, we can focus on
the \textit{actual abstract} object struct v.s. the \textit{possible system} that have the structure.
In this case, \textit{actual abstract} implies epistemic access,
and \textit{possible system} shared uniform semantics.
Therefore, we still face the problems in Benacerraf's dilemma.

\end{document}